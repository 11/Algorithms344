
\documentclass{article}

\usepackage{fancyhdr}
\usepackage{enumerate}
\usepackage{amsmath}
\usepackage{amsfonts}
\usepackage{amssymb}

\usepackage{tikz}
\usetikzlibrary{arrows,automata}
\usepackage{graphicx}
\usepackage{grffile}

%
% Basic Document Settings
%

\topmargin=-0.45in
\evensidemargin=0in
\oddsidemargin=0in
\textwidth=6.5in
\textheight=9.0in
\headsep=0.25in

\linespread{1.1}

\pagestyle{fancy}
\lhead{\hmwkAuthorName}
\rhead{\hmwkClass: \hmwkTitle}
\cfoot{\thepage}

\renewcommand\headrulewidth{0.4pt}
\renewcommand\footrulewidth{0.4pt}

\setlength\parindent{0pt}

%
% Homework Details
%   - Title
%   - Due date
%   - Class
%   - Section/Time
%   - Instructor
%   - Author
%

\newcommand{\hmwkTitle}{Homework\ \#4}
\newcommand{\hmwkDueDate}{April 12, 2017}
\newcommand{\hmwkClass}{Design \& Analysis of Algorithms - 344}
\newcommand{\hmwkClassTime}{Section \#2}
\newcommand{\hmwkClassInstructor}{Professor Dr. Kalantari}
\newcommand{\hmwkAuthorName}{\textbf{Douglas Rudolph, Mustufa Hussain}}

%
% Title Page
%

\title{
    \vspace{2in}
    \textmd{\textbf{\hmwkClass:\ \hmwkTitle}}\\
    \normalsize\vspace{0.1in}\small{Due\ on\ \hmwkDueDate\ at 11:59pm}\\
    \vspace{0.1in}\large{\textit{\hmwkClassInstructor\ \hmwkClassTime}}
    \vspace{3in}
}

\author{\hmwkAuthorName}
\date{}

\setlength\parskip{\baselineskip}

\begin{document}
\maketitle
\pagebreak


\textbf{Problem 2}. 
Given 4 matrices, $A1, . . . , A4$, where $Ai$ is $m_{i - 1}$ x $m_{i}$ matrix, $i = 1, . . . , 4$, with
$m_{0} = 50$, $m_{1} = 10$, $m_{2} = 30$, $m_{3} = 20$, $m_{4} = 100$. Compute c(i, j)’s by following the matrix chain multiplication algorithm.\\

When running the matrix chain multiplication algorithm, the goal is to find the best order to multiply some some set of matrices $A_{1} * A_{2} * ... * A_{n}$ for a given value $n$. To accomplish this, we begin by finding the fastest way multiply any 2 matrices that are adjacent to one another and store the result in some type of data structure. After finding the amount of multiplications it would take to multiply any two adjacent matrices, we repeat the same process - but for three adjacent matrices. This process is repeated recursively up until we calculate the most efficient way to multiply a set of $n$ adjacent matrices. 

To start let $L = 1$ where $L$ is a variable that refers to the amount adjacent matrices we are multiplying together. Also, let the matrix below be the data structure showing the progression of calculating each matrix for each recursive iteration of the matrix chain algorithm. 

In the case that \textbf{L = 1}, the output is as follows. The data in the matrix is showing the amount of multiplications it would take to multiply a given matrix with nothing. 

\begin{center}
	\begin{tabular}{| c c c |}
		
		\hline
		$A_{1} * 0$ &$=$ &$0$\\
		\hline
		$A_{2} * 0$ &$=$ &$0$ \\
		\hline
		$A_{3} * 0$ &$=$ &$0$ \\
		\hline
	
	\end{tabular}
\end{center}

\begin{center}
	\begin{tabular}{c | c | c | c | c}
	
		&$1$ &$2$ &$3$ &$4$ \\
		\hline
		$1$ &$0$ & & &\\
		\hline
		$2$ & &$0$ & &\\
		\hline
		$3$ & & &$0$ &\\
		\hline
		$4$ & & & &$0$\\
		
	\end{tabular}
\end{center}

Now let \textbf{L = 2}. At this step we have to record the amount of multiplications it would take to multiply 2 adjacent matrices. The set of multiplications would be: $M = \{(A_{1} * A_{2}), (A_{2} * A_{3}), (A_{3} * A_{4})\}$. 

\begin{center}
	\begin{tabular}{| c c c c c |}
		\hline
		$A_{1} * A_{2}$ &$=$ &$50 * 10 * 30$ &$=$ &$15,000$\\
		\hline		
		$A_{2} * A_{3}$ &$=$ &$10 * 30 * 20$ &$=$ &$6,000$ \\
		\hline
		$A_{3} * A_{4}$ &$=$ &$30 * 20 * 100$ &$=$ &$60,000$\\ 
		\hline	
	
	\end{tabular}
\end{center}
\begin{center}

	\begin{tabular}{c | c | c | c | c}
	
		&$1$ &$2$ &$3$ &$4$ \\
		\hline
		$1$ &$0$ &$15,000$ & &\\
		\hline
		$2$ & &$0$ &$6,000$ &\\
		\hline
		$3$ & & &$0$ &$60,000$\\
		\hline
		$4$ & & & &$0$\\
		
	\end{tabular}
\end{center}

Now let \textbf{L = 3}. At this step we have to record the amount of multiplications it would take to multiply 3 adjacent matrices. The set of multiplications would be: $M = \{(A_{1})(A_{2} * A_{3}), (A_{1} *A_{2})(A_{3}), (A_{2} * A_{3})(A_{4}), (A_{2})(A_{3} * A_{4})\}$. (\textbf{NOTE}: notice how 1 of the 2 calculations that are being multiplied at this step was already calculated and stored within the matrix from the previous iteration of the matrix chain algorithm.)

\begin{center}
	\begin{tabular}{|c c c c c |}
		
		\hline
		$(A_{1})(A_{2} * A_{3})$ &$=$ &$50 * 10 * 20 + 6000$ &$=$ &$16,000$\\
		$(A_{1} *A_{2})(A_{3})$ &$=$ &$15,000 + 50 * 30 * 20$ &$=$ &$45,000$ \\
		\hline
		$(A_{2} * A_{3})(A_{4})$ &$=$ &$6,000 + 10 * 20 * 100$ &$=$ &$26,000$ \\
		$(A_{2})(A_{3} * A_{4})$ &$=$ &$10 * 30 * 100 + 60,000$ &$=$ &$90,000$ \\
		\hline
	
	\end{tabular}
\end{center}
\begin{center}

	\begin{tabular}{c | c | c | c | c}
	
		&$1$ &$2$ &$3$ &$4$ \\
		\hline
		$1$ &$0$ &$15,000$ &$16,000$ &\\
		\hline
		$2$ & &$0$ &$6,000$ &$26,000$\\
		\hline
		$3$ & & &$0$ &$60,000$\\
		\hline
		$4$ & & & &$0$\\
		
	\end{tabular}
\end{center}

Now let \textbf{L = 4}. At this step we have to record the amount of multiplications it would take to multiply 4 adjacent matrices. The set of multiplications would be: $M = \{(A_{1})(A_{2} * A_{3}*A_{4}), (A_{1} *A_{2})(A_{3}*A_{4}), (A_{1} * A_{2} * A_{3})(A_{4})\}$. 

\begin{center}
	\begin{tabular}{| c c c c c |}
		
		\hline	
		$(A_{1})(A_{2} * A_{3}*A_{4})$   &$=$ &$50 * 10 * 100 + 26,000$ &$=$ &$76,000$\\
		\hline		
		$(A_{1} *A_{2})(A_{3}*A_{4})$    &$=$ &$15,000 + 60,000 + 50 * 10 * 100$ &$=$ &$175,000$ \\
		\hline		
		$(A_{1} * A_{2} * A_{3})(A_{4})$ &$=$ &$16,000 + 50 * 20 * 100$ &$=$ &$116,000$ \\
		\hline
		
	\end{tabular}
\end{center}
\begin{center}

	\begin{tabular}{c | c | c | c | c}
	
		&$1$ &$2$ &$3$ &$4$ \\
		\hline
		$1$ &$0$ &$15,000$ &$16,000$ &$76,000$\\
		\hline
		$2$ & &$0$ &$6,000$ &$26,000$\\
		\hline
		$3$ & & &$0$ &$60,000$\\
		\hline
		$4$ & & & &$0$\\
		
	\end{tabular}
\end{center}

After running through the algorithm and recursively defining matrix multiplication by calculating \textit{min( c(i-1, j-1)+c(i,j) )} at each step, we are able to then calculate all \textit{c(i,j)s}

\newpage
\end{document}